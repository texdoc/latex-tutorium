%       $Id: fortgeschrittenes.tex,v 1.11 2005/06/28 18:30:35 bronger Exp $    
%
%     fortgeschrittenes.tex -- Part of the LaTeX Tutorium
%     Copyright 2004 Project Members of
%                    http://sourceforge.net/projects/latex-tutorium/
%                    
%
%   This program is free software; you can redistribute it and/or
%   modify it under the terms of the Artistic License 2.0 as published
%   by Larry Wall.  You should have received a copy of the Artistic
%   License 2.0 along with this program in the file COPYING; if not,
%   you can get it at
%     http://dev.perl.org/rfc/346.html
%   or contact the current maintainers of the LaTeX Tutorium.
%
%   This program is distributed in the hope that it will be useful, but
%   WITHOUT ANY WARRANTY; without even the implied warranty of
%   MERCHANTABILITY or FITNESS FOR A PARTICULAR PURPOSE.  See the
%   Artistic License 2.0 for more details.
%
%   This file may only be distributed together with a copy of the LaTeX
%   Turorium.
%
%   The LaTeX Tutorium consists of all files listed in manifest.txt.


\section{Fortgeschrittenes Layout}

\minisec{Fu�noten}

Fu�noten werden mit \Alt+\Ctrl+\keystroke F erzeugt.  Diese Tastenkombination
f�gt den Befehl
\begin{lstlisting}
\footnote{}
\end{lstlisting}
ein.  Zwischen die geschweiften Klammern schreibt man dann den Text, der zur
Fu�note werden soll:\footnote{So wie hier}
\begin{lstlisting}
... werden soll:\footnote{So wie hier}
\end{lstlisting}

\minisec{Links ins Internet}

Internet-Adressen schreibt man mit dem Befehl \lstinline{\url}, der vom Paket
\Package{url} bereit gestellt wird (die Einbindung erkl�rt Abschnitt
\vref{sec:pakete}).
\begin{lstlisting}
Die Homepage der RWTH Aachen ist
\url{http://www.rwth-aachen.de}.
\end{lstlisting}
Im \PDF-Dokument sind diese Links nat�rlich anklickbar.


\subsection{Seitenr�nder einstellen}

\begin{table}
  \caption{Optionen zum Einstellen der Seitenr�nder}
  \label{tab:Seitenraender}
  \vspace{1ex}
  \centering
  \begin{tabular}{@{}ll@{}}
    \toprule
    K�rzel  &  Beschreibung \\
    \midrule
    a$5$paper      &  Papierformat auf A$5$ einstellen \\
    left=$X$cm     &  linken Rand auf $X$\,cm einstellen \\
    right=$X$cm    &  rechten Rand auf $X$\,cm einstellen \\
    top=$X$cm      &  oberen Rand auf $X$\,cm einstellen \\
    bottom=$X$cm   &  unteren Rand auf $X$\,cm einstellen \\
    \bottomrule
  \end{tabular}
\end{table}

Seitenr�nder werden in \LaTeX{} am besten mit dem Befehl
\begin{lstlisting}
\usepackage[...]{geometry}
\end{lstlisting}
eingestellt.  Dieser Befehl muss irgendwo in der Pr�ambel vor dem
\lstinline|\begin{document}| kommen.  Zwischen die eckigen Klammern schreibt
  man dann, wie man die Seitenr�nder gerne h�tte, zum Beispiel:
\begin{lstlisting}
\usepackage[a5paper,
  left=1.9cm, right=2.1cm,
  top=1.2cm, bottom=2.3cm]{geometry}
\end{lstlisting}
Das stellt auf \versalien{DIN}\,A$5$-Papier ein, und setzt die vier R�nder auf
die angegebenen Werte in Zenimentern.  Die Tabelle~\vref{tab:Seitenraender}
listet die wichtigsten Optionen f�r die Seiten-Einstellung auf.  (Es gibt
allerdings noch mehr, z.\,B. f�r Kopf- und Fu�zeilen.)


\subsection{Andere Sprachen}

\begin{table}
  \caption{Beispiele f�r Sprachen bei \LaTeX{}}
  \label{tab:Sprachen}
  \vspace{1ex}
  \centering
  \begin{tabular}{@{}ll@{}}
    \toprule
    english             &  amerikanisches Englisch \emph{(Voreinstellung)}  \\
    UKenglish           &  britisches Englisch                              \\
    german              &  Deutsch, Alte Rechtschreibung                    \\
    ngerman             &  Deutsch, Neue Rechtschreibung                    \\
    french              &  Franz�sisch                                      \\
    \bottomrule
  \end{tabular}
\end{table}

Wenn man nichts anderes sagt, geht \LaTeX{} davon aus, dass man einen Text auf
Englisch schreibt.  Werfen wir nochmal einen Blick auf das Beispiel auf
Seite~\pageref{lst:ngerman}, und zwar auf die Zeile~\ref{lst:ngerman}:
\begin{lstlisting}
\usepackage[ngerman]{babel}
\end{lstlisting}
Diese Zeile stellt die Sprache auf "`Deutsch -- Neue Rechtschreibung"' ein.
Falls man statt "`ngerman"' einfach nur "`german"' schreibt, werden alte
Rechtschreibregeln angewandt.  Tabelle~\vref{tab:Sprachen} listet ein paar
Sprachk�rzel auf, die man zwischen die eckigen Klammern schreiben kann.  Aber
es gibt ebenso "`italian"', "`dutch"', "`spanish"' und noch viele weitere, auch
f�r nicht-lateinische Schriften.

�brigens hat diese Einstellung nur Auswirkungen auf den \emph{Druck} des
Textes, insbesondere auf die korrekte Silbentrennung.  Die
\emph{Rechtschreibkontrolle} ist eine davon unabh�ngige Funktion.


%%% Local Variables: 
%%% mode: latex
%%% TeX-master: "latex-tutorium"
%%% End: 
