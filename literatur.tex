%       $Id: literatur.tex,v 1.7 2004/05/23 10:56:44 bronger Exp $    
%
%     literatur.tex -- Part of the LaTeX Tutorium
%     Copyright 2004 Project Members of
%                    https://github.com/latextemplates/latex-tutorium/
%                    
%
%   This program is free software; you can redistribute it and/or
%   modify it under the terms of the Artistic License 2.0 as published
%   by Larry Wall.  You should have received a copy of the Artistic
%   License 2.0 along with this program in the file COPYING; if not,
%   you can get it at
%     http://dev.perl.org/rfc/346.html
%   or contact the current maintainers of the LaTeX Tutorium.
%
%   This program is distributed in the hope that it will be useful, but
%   WITHOUT ANY WARRANTY; without even the implied warranty of
%   MERCHANTABILITY or FITNESS FOR A PARTICULAR PURPOSE.  See the
%   Artistic License 2.0 for more details.
%
%   This file may only be distributed together with a copy of the LaTeX
%   Turorium.
%
%   The LaTeX Tutorium consists of all files listed in manifest.txt.


\setbibpreamble{Diese kleine Einf�hrung ist damit zuende.  Nat�rlich kann
  \LaTeX{} noch viel viel mehr, wie man schon am Layout dieses Tutoriums sehen
  kann.  Daher folgt nun eine knappe Auf"|listung weitergehender
  \LaTeX-Literatur, die wir gut finden -- mit einer Ausnahme.  Die B�cher sind
  von "`neu"' nach "`alt"' sortiert.  Einige kann man im Internet
  herunterladen.
  
  Weitere Hilfe findet man auch bei der Deutschen Anwendervereinigung \TeX{}
  \versalien{DANTE}~e.V. `\url{www.dante.de}', und in der deutschen
  Usenet-\TeX-Gruppe bei
  `\url{groups.google.com/groups?group=de.comp.text.tex&hl=de}'.\par
  \raggedright\nonfrenchspacing}

\renewcommand{\refname}{Weiterf�hrende Literatur\label{sec:literatur}}

\nocite{*}
\bibliographystyle{latex-tutorium}
\bibliography{latex-tutorium}

%%% Local Variables: 
%%% mode: latex
%%% TeX-master: "latex-tutorium"
%%% End: 
