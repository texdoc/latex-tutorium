%       $Id: freepdf.tex,v 1.1 2005/06/28 20:15:36 bronger Exp $    
%
%     freepdf.tex -- Part of the LaTeX Tutorium
%     Copyright 2004 Project Members of
%                    https://github.com/latextemplates/latex-tutorium/
%                    
%
%   This program is free software; you can redistribute it and/or
%   modify it under the terms of the Artistic License 2.0 as published
%   by Larry Wall.  You should have received a copy of the Artistic
%   License 2.0 along with this program in the file COPYING; if not,
%   you can get it at
%     http://dev.perl.org/rfc/346.html
%   or contact the current maintainers of the LaTeX Tutorium.
%
%   This program is distributed in the hope that it will be useful, but
%   WITHOUT ANY WARRANTY; without even the implied warranty of
%   MERCHANTABILITY or FITNESS FOR A PARTICULAR PURPOSE.  See the
%   Artistic License 2.0 for more details.
%
%   This file may only be distributed together with a copy of the LaTeX
%   Turorium.
%
%   The LaTeX Tutorium consists of all files listed in manifest.txt.


\section{Benutzung von FreePDF}
\label{sec:FreePDF-usage}

Wie schon erw�hnt, sind \PDF-Dateien die beste M�glichkeit, Diagramme,
schematische Zeichnungen und Plots in den eigenen Text zu bekommen.  Nur wie
bekomme ich \PDF-Dateien?

FreePDF macht die Erzeugung von \PDF-Dateien zum Kinderspiel.  Man muss
lediglich die Abbildung, die man in den eigenen \LaTeX-Text einbinden m�chte,
\emph{ausdrucken}.  Der Clou ist, dass man die Datei nat�rlich nicht auf einem
richtigen Drucker ausdruckt (davon h�tte man ja nichts), sondern als
Drucker-Namen "`FreePDF"' ausw�hlt.

Danach kann man sich noch aussuchen, wohin die \PDF-Datei gespeichert werden
soll.  Dabei w�hlt man "`Ablegen"' und hangelt sich bis zu dem Ordner durch, in
dem der \LaTeX-Text liegt -- fertig.


%%% Local Variables: 
%%% mode: latex
%%% TeX-master: "latex-tutorium"
%%% End: 
